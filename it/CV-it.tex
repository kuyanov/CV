\documentclass[letterpaper,11pt]{article}

\usepackage{latexsym}
\usepackage[empty]{fullpage}
\usepackage{titlesec}
\usepackage{marvosym}
\usepackage[usenames,dvipsnames]{color}
\usepackage{verbatim}
\usepackage{enumitem}
\usepackage[hidelinks]{hyperref}
\usepackage{fancyhdr}
\usepackage[english]{babel}
\usepackage{tabularx}
\usepackage{fontawesome5}
\usepackage{multicol}
\setlength{\multicolsep}{-3.0pt}
\setlength{\columnsep}{-1pt}
\input{glyphtounicode}


%----------FONT OPTIONS----------
% sans-serif
% \usepackage[sfdefault]{FiraSans}
% \usepackage[sfdefault]{roboto}
% \usepackage[sfdefault]{noto-sans}
% \usepackage[default]{sourcesanspro}

% serif
% \usepackage{CormorantGaramond}
% \usepackage{charter}


\pagestyle{fancy}
\fancyhf{} % clear all header and footer fields
\fancyfoot{}
\renewcommand{\headrulewidth}{0pt}
\renewcommand{\footrulewidth}{0pt}

\addtolength{\oddsidemargin}{-0.6in}
\addtolength{\evensidemargin}{-0.5in}
\addtolength{\textheight}{1.4in}
\addtolength{\textwidth}{1.19in}
\addtolength{\topmargin}{-0.7in}

\urlstyle{same}

\raggedbottom
\raggedright
\setlength{\tabcolsep}{0in}

\titleformat{\section}{
    \scshape\raggedright\large\bfseries
}{}{0em}{}[\color{black}\titlerule \vspace{-6pt}]

\pdfgentounicode=1

\newcommand{\resumeSubHeading}[1]{
    \item\textbf{#1}\vspace{-7pt}
}

\newcommand{\resumeTwoLineHeading}[4]{
    \item
    \begin{tabular*}{1.0\textwidth}[t]{l@{\extracolsep{\fill}}r}
        \textbf{#1} & \textbf{\small #2} \\
        \textit{\small #3} & \textit{\small #4} \\
    \end{tabular*}\vspace{-7pt}
}

\newcommand{\resumeProjectHeading}[2]{
    \item
    \begin{tabular*}{1.0\textwidth}{l@{\extracolsep{\fill}}r}
        \small #1 & \textbf{\small #2} \\
    \end{tabular*}\vspace{-7pt}
}

\newcommand{\resumeSubHeadingListStart}{\begin{itemize}[leftmargin=0.0in, label={}]}
\newcommand{\resumeSubHeadingListEnd}{\end{itemize}}
\newcommand{\resumeItemListStart}{\begin{itemize}}
\newcommand{\resumeItemListEnd}{\end{itemize}\vspace{-8pt}}
\newcommand{\resumeItem}[1]{\item\small{{#1 \vspace{-2pt}}}}
\newcommand{\resumeSubItem}[1]{\resumeItem{#1}\vspace{-4pt}}

\renewcommand\labelitemi{$\vcenter{\hbox{\tiny$\bullet$}}$}
\renewcommand\labelitemii{$\vcenter{\hbox{\tiny$\bullet$}}$}


\begin{document}


\begin{center}
    {\Huge \scshape Fedor Kuyanov} \\ \vspace{3pt}
    Moscow, Russia \\ \vspace{5pt}
    \small \raisebox{-0.1\height}\faPhone\ +7 (926) 780-75-40 ~ \href{mailto:feodor.kuyanov@gmail.com}{\raisebox{-0.2\height}\faEnvelope\  \underline{feodor.kuyanov@gmail.com}} ~ 
    \href{https://github.com/kuyanov}{\raisebox{-0.2\height}\faGithub\ \underline{github.com/kuyanov}}
    \vspace{-8pt}
\end{center}


\section{Education}
    \resumeSubHeadingListStart
        \resumeTwoLineHeading
        {Higher School of Economics}{September 2020 -- July 2024}
        {Bachelor in Computer Science, GPA 9.81/10}{Moscow, Russia}
    \resumeSubHeadingListEnd


\section{Work experience}
    \resumeSubHeadingListStart
        \resumeTwoLineHeading
        {Yandex}{May 2021 -- August 2021}
        {Software Engineer Intern}{Moscow, Russia}
        \resumeItemListStart
            \resumeItem{Member of the team responsible for the quality of Yandex voice assistant Alice.}
            \resumeItem{Enhanced scenario classification by adding new factors and adapting learning process for different devices.}
            \resumeItem{Added tests to ensure classifier's validity.}
            \resumeItem{Ran an experiment on real users to measure its quality.}
            \resumeItem{As a result, the classifier's quality had been improved on some devices by 30\%.}
        \resumeItemListEnd
    \resumeSubHeadingListEnd


\section{Projects}
    \resumeSubHeadingListStart

        \resumeProjectHeading
        {\textbf{\href{https://github.com/m20-sch57/fejudge}{\underline{Fejudge}}} $|$ \emph{Python, Linux kernel}}{September 2021}
        \resumeItemListStart
            \resumeItem{This is a contest management system, like Topcoder or Codeforces.}
            \resumeItem{It uses several Linux kernel features and supports simultaneous evaluation of solutions on different machines.}
            \resumeItem{Also provides a nice web interface for participants.}
        \resumeItemListEnd

        \resumeProjectHeading
        {\textbf{\href{https://github.com/m20-sch57/thetruehat}{\underline{TheTrueHat}}} $|$ \emph{JavaScript}}{August 2021}
        \resumeItemListStart
            \resumeItem{This project allows you to play the Hat (Alias) game from your browser.}
        \resumeItemListEnd 

        \resumeProjectHeading
        {\textbf{\href{https://github.com/JaggedLine/kotline}{\underline{Kotline}}} $|$ \emph{Kotlin}}{January 2021}
        \resumeItemListStart
            \resumeItem{This is a geometric puzzle, in which you must build the longest non-self-intersecting polygonal chain from start to end.}
            \resumeItem{Also contains a mode which allows you to find the best solution automatically.}
        \resumeItemListEnd

        \resumeProjectHeading
        {\textbf{\href{https://github.com/m20-sch57/exam-system}{\underline{Examiner}}} $|$ \emph{Python, QT}}{September 2019}
        \resumeItemListStart
            \resumeItem{The purpose of this project is to build a platform to automate school exams.}
            \resumeItem{Provides UIs for students and teachers and enables automatic verification of various tasks.}
        \resumeItemListEnd

    \resumeSubHeadingListEnd


\section{Skills \& interests}
    \resumeItemListStart
        \resumeSubItem{\textbf{Programming languages: }{C/C++, Python, Kotlin, HTML/CSS, JavaScript, Bash, SQL}}
        \resumeSubItem{\textbf{Libraries \& frameworks: }{QT, numpy, pytorch, Flask, React, Vue}}
        \resumeSubItem{\textbf{Tools: }{CI/CD, git, docker, gdb, cmake}}
        \resumeSubItem{\textbf{Computer science skills: }{advanced algorithms \& data structures -- studied since 2016 for programming competitions}}
        \resumeSubItem{\textbf{Mathematical skills: }{linear algebra and geometry, discrete math, probability theory, math analysis}}
        \resumeSubItem{\textbf{Interested in: }{theoretical computer science, high-load systems, web-servers, front-end (a little)}}
    \resumeItemListEnd


\section{Olympiad achievements}
    \resumeItemListStart
        \resumeSubItem{Gold medal at \href{https://imc-math.org.uk/?year=2021}{\underline{IMC 2021}}}
        \resumeSubItem{Silver medal at \href{https://neerc.ifmo.ru/archive/2020.html}{\underline{Semifinal ICPC 2020, NERC}}}
        \resumeSubItem{1st place at the All-Russian Team Olympiad in Informatics (2018-2019)}
        \resumeSubItem{Winner of the All-Russian Olympiad in Informatics (2018)}
        \resumeSubItem{\href{https://codeforces.com/profile/Kuyan}{\underline{Codeforces}}}
    \resumeItemListEnd


\section{Languages}
    \resumeItemListStart
        \resumeSubItem{English: C1 (IELTS 7.0)}
        \resumeSubItem{Russian: Native speaker}
    \resumeItemListEnd


\section{Hobbies}
    \resumeItemListStart
        \resumeSubItem{Railway modelling, Lego}
        \resumeSubItem{Chess, speed-cubing (3x3, 36 secs), ping-pong}
        \resumeSubItem{Classical piano -- finished 8-year Gnesin music school and performed with the orchestra, \href{https://www.youtube.com/user/FeodorKuyanov/playlists}{\underline{YouTube}}, winner of Moscow and international piano competitions}
    \resumeItemListEnd


\end{document}
